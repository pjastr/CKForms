% Options for packages loaded elsewhere
\PassOptionsToPackage{unicode}{hyperref}
\PassOptionsToPackage{hyphens}{url}
%
\documentclass[
]{article}
\usepackage{amsmath,amssymb}
\usepackage{lmodern}
\usepackage{ifxetex,ifluatex}
\ifnum 0\ifxetex 1\fi\ifluatex 1\fi=0 % if pdftex
  \usepackage[T1]{fontenc}
  \usepackage[utf8]{inputenc}
  \usepackage{textcomp} % provide euro and other symbols
\else % if luatex or xetex
  \usepackage{unicode-math}
  \defaultfontfeatures{Scale=MatchLowercase}
  \defaultfontfeatures[\rmfamily]{Ligatures=TeX,Scale=1}
\fi
% Use upquote if available, for straight quotes in verbatim environments
\IfFileExists{upquote.sty}{\usepackage{upquote}}{}
\IfFileExists{microtype.sty}{% use microtype if available
  \usepackage[]{microtype}
  \UseMicrotypeSet[protrusion]{basicmath} % disable protrusion for tt fonts
}{}
\makeatletter
\@ifundefined{KOMAClassName}{% if non-KOMA class
  \IfFileExists{parskip.sty}{%
    \usepackage{parskip}
  }{% else
    \setlength{\parindent}{0pt}
    \setlength{\parskip}{6pt plus 2pt minus 1pt}}
}{% if KOMA class
  \KOMAoptions{parskip=half}}
\makeatother
\usepackage{xcolor}
\IfFileExists{xurl.sty}{\usepackage{xurl}}{} % add URL line breaks if available
\IfFileExists{bookmark.sty}{\usepackage{bookmark}}{\usepackage{hyperref}}
\hypersetup{
  pdftitle={CKForms - Manual},
  hidelinks,
  pdfcreator={LaTeX via pandoc}}
\urlstyle{same} % disable monospaced font for URLs
\usepackage[margin=1in]{geometry}
\usepackage{color}
\usepackage{fancyvrb}
\newcommand{\VerbBar}{|}
\newcommand{\VERB}{\Verb[commandchars=\\\{\}]}
\DefineVerbatimEnvironment{Highlighting}{Verbatim}{commandchars=\\\{\}}
% Add ',fontsize=\small' for more characters per line
\usepackage{framed}
\definecolor{shadecolor}{RGB}{248,248,248}
\newenvironment{Shaded}{\begin{snugshade}}{\end{snugshade}}
\newcommand{\AlertTok}[1]{\textcolor[rgb]{0.94,0.16,0.16}{#1}}
\newcommand{\AnnotationTok}[1]{\textcolor[rgb]{0.56,0.35,0.01}{\textbf{\textit{#1}}}}
\newcommand{\AttributeTok}[1]{\textcolor[rgb]{0.77,0.63,0.00}{#1}}
\newcommand{\BaseNTok}[1]{\textcolor[rgb]{0.00,0.00,0.81}{#1}}
\newcommand{\BuiltInTok}[1]{#1}
\newcommand{\CharTok}[1]{\textcolor[rgb]{0.31,0.60,0.02}{#1}}
\newcommand{\CommentTok}[1]{\textcolor[rgb]{0.56,0.35,0.01}{\textit{#1}}}
\newcommand{\CommentVarTok}[1]{\textcolor[rgb]{0.56,0.35,0.01}{\textbf{\textit{#1}}}}
\newcommand{\ConstantTok}[1]{\textcolor[rgb]{0.00,0.00,0.00}{#1}}
\newcommand{\ControlFlowTok}[1]{\textcolor[rgb]{0.13,0.29,0.53}{\textbf{#1}}}
\newcommand{\DataTypeTok}[1]{\textcolor[rgb]{0.13,0.29,0.53}{#1}}
\newcommand{\DecValTok}[1]{\textcolor[rgb]{0.00,0.00,0.81}{#1}}
\newcommand{\DocumentationTok}[1]{\textcolor[rgb]{0.56,0.35,0.01}{\textbf{\textit{#1}}}}
\newcommand{\ErrorTok}[1]{\textcolor[rgb]{0.64,0.00,0.00}{\textbf{#1}}}
\newcommand{\ExtensionTok}[1]{#1}
\newcommand{\FloatTok}[1]{\textcolor[rgb]{0.00,0.00,0.81}{#1}}
\newcommand{\FunctionTok}[1]{\textcolor[rgb]{0.00,0.00,0.00}{#1}}
\newcommand{\ImportTok}[1]{#1}
\newcommand{\InformationTok}[1]{\textcolor[rgb]{0.56,0.35,0.01}{\textbf{\textit{#1}}}}
\newcommand{\KeywordTok}[1]{\textcolor[rgb]{0.13,0.29,0.53}{\textbf{#1}}}
\newcommand{\NormalTok}[1]{#1}
\newcommand{\OperatorTok}[1]{\textcolor[rgb]{0.81,0.36,0.00}{\textbf{#1}}}
\newcommand{\OtherTok}[1]{\textcolor[rgb]{0.56,0.35,0.01}{#1}}
\newcommand{\PreprocessorTok}[1]{\textcolor[rgb]{0.56,0.35,0.01}{\textit{#1}}}
\newcommand{\RegionMarkerTok}[1]{#1}
\newcommand{\SpecialCharTok}[1]{\textcolor[rgb]{0.00,0.00,0.00}{#1}}
\newcommand{\SpecialStringTok}[1]{\textcolor[rgb]{0.31,0.60,0.02}{#1}}
\newcommand{\StringTok}[1]{\textcolor[rgb]{0.31,0.60,0.02}{#1}}
\newcommand{\VariableTok}[1]{\textcolor[rgb]{0.00,0.00,0.00}{#1}}
\newcommand{\VerbatimStringTok}[1]{\textcolor[rgb]{0.31,0.60,0.02}{#1}}
\newcommand{\WarningTok}[1]{\textcolor[rgb]{0.56,0.35,0.01}{\textbf{\textit{#1}}}}
\usepackage{graphicx}
\makeatletter
\def\maxwidth{\ifdim\Gin@nat@width>\linewidth\linewidth\else\Gin@nat@width\fi}
\def\maxheight{\ifdim\Gin@nat@height>\textheight\textheight\else\Gin@nat@height\fi}
\makeatother
% Scale images if necessary, so that they will not overflow the page
% margins by default, and it is still possible to overwrite the defaults
% using explicit options in \includegraphics[width, height, ...]{}
\setkeys{Gin}{width=\maxwidth,height=\maxheight,keepaspectratio}
% Set default figure placement to htbp
\makeatletter
\def\fps@figure{htbp}
\makeatother
\setlength{\emergencystretch}{3em} % prevent overfull lines
\providecommand{\tightlist}{%
  \setlength{\itemsep}{0pt}\setlength{\parskip}{0pt}}
\setcounter{secnumdepth}{-\maxdimen} % remove section numbering
\ifluatex
  \usepackage{selnolig}  % disable illegal ligatures
\fi
\newlength{\cslhangindent}
\setlength{\cslhangindent}{1.5em}
\newlength{\csllabelwidth}
\setlength{\csllabelwidth}{3em}
\newenvironment{CSLReferences}[2] % #1 hanging-ident, #2 entry spacing
 {% don't indent paragraphs
  \setlength{\parindent}{0pt}
  % turn on hanging indent if param 1 is 1
  \ifodd #1 \everypar{\setlength{\hangindent}{\cslhangindent}}\ignorespaces\fi
  % set entry spacing
  \ifnum #2 > 0
  \setlength{\parskip}{#2\baselineskip}
  \fi
 }%
 {}
\usepackage{calc}
\newcommand{\CSLBlock}[1]{#1\hfill\break}
\newcommand{\CSLLeftMargin}[1]{\parbox[t]{\csllabelwidth}{#1}}
\newcommand{\CSLRightInline}[1]{\parbox[t]{\linewidth - \csllabelwidth}{#1}\break}
\newcommand{\CSLIndent}[1]{\hspace{\cslhangindent}#1}

\title{CKForms - Manual}
\author{}
\date{\vspace{-2.5em}}

\begin{document}
\maketitle

\hypertarget{notation-and-convention}{%
\subsection{Notation and convention}\label{notation-and-convention}}

In this manual we use small latin leters \texttt{g},\texttt{h},\ldots{}
or small gothic letters \(\mathfrak{g}\), \(\mathfrak{h}\),\ldots{} for
Lie algebras.

We use the notation of the theory of real Lie algebra from CoReLG
Package, (\protect\hyperlink{ref-CoReLG}{Dietrich, Faccin, and Graaf
2014}). Every real simple Lie algebra is identified with triple
\texttt{(type,rank,\ id)}. Be careful, \texttt{rank} is not always
considered as the dimension of the Cartan subalgebra of the simple Lie
algebra \(\mathfrak{g}\). It is equal to the dimension of the Cartan
subalgebra for all real simple Lie algebras excluding realifications
(that is, complex simple lie algebras considered as real Lie algebras).
For realifications \texttt{id} is equal to zero and \texttt{rank} is
equal to half of the rank of the Lie algebra \(\mathfrak{g}\). Consider
the following example.

\begin{Shaded}
\begin{Highlighting}[]
\ExtensionTok{gap}\OperatorTok{\textgreater{}}\NormalTok{ RealFormsInformation}\ErrorTok{(}\StringTok{"E"}\ExtensionTok{,6}\KeywordTok{);}

  \ExtensionTok{There}\NormalTok{ are 5 simple real forms with complexification E6}
    \ExtensionTok{1}\NormalTok{ is the compact form}
    \ExtensionTok{2}\NormalTok{ is EI   = E6}\ErrorTok{(}\ExtensionTok{6}\KeywordTok{)}\ExtensionTok{,}\NormalTok{ with k\_0 of type sp}\ErrorTok{(}\ExtensionTok{4}\KeywordTok{)} \KeywordTok{(}\ExtensionTok{C4}\KeywordTok{)}
    \ExtensionTok{3}\NormalTok{ is EII  = E6}\ErrorTok{(}\ExtensionTok{2}\KeywordTok{)}\ExtensionTok{,}\NormalTok{ with k\_0 of type su}\ErrorTok{(}\ExtensionTok{6}\KeywordTok{)}\ExtensionTok{+su}\ErrorTok{(}\ExtensionTok{2}\KeywordTok{)} \KeywordTok{(}\ExtensionTok{A5+A1}\KeywordTok{)}
    \ExtensionTok{4}\NormalTok{ is EIII = E6}\ErrorTok{(}\ExtensionTok{{-}14}\KeywordTok{)}\ExtensionTok{,}\NormalTok{ with k\_0 of type so}\ErrorTok{(}\ExtensionTok{10}\KeywordTok{)}\ExtensionTok{+R} \ErrorTok{(}\ExtensionTok{D5+R}\KeywordTok{)}
    \ExtensionTok{5}\NormalTok{ is EIV  = E6}\ErrorTok{(}\ExtensionTok{{-}26}\KeywordTok{)}\ExtensionTok{,}\NormalTok{ with k\_0 of type f\_4 }\ErrorTok{(}\ExtensionTok{F4}\KeywordTok{)}
  \ExtensionTok{Index} \StringTok{\textquotesingle{}0\textquotesingle{}}\NormalTok{ returns the realification of E6}

\ExtensionTok{gap}\OperatorTok{\textgreater{}}\NormalTok{ g:=RealFormById}\ErrorTok{(}\StringTok{"E"}\ExtensionTok{,6,2}\KeywordTok{);}
\OperatorTok{\textless{}}\NormalTok{Lie }\ExtensionTok{algebra}\NormalTok{ of dimension 78 over SqrtField}\OperatorTok{\textgreater{}}
\ExtensionTok{gap}\OperatorTok{\textgreater{}}\NormalTok{ Dimension}\ErrorTok{(}\ExtensionTok{CartanSubalgebra}\ErrorTok{(}\ExtensionTok{g}\KeywordTok{));}
\ExtensionTok{6}
\ExtensionTok{gap}\OperatorTok{\textgreater{}}\NormalTok{ h:=RealFormById}\ErrorTok{(}\StringTok{"E"}\ExtensionTok{,6,0}\KeywordTok{);}
\OperatorTok{\textless{}}\NormalTok{Lie }\ExtensionTok{algebra}\NormalTok{ of dimension 156 over SqrtField}\OperatorTok{\textgreater{}}
\ExtensionTok{gap}\OperatorTok{\textgreater{}}\NormalTok{ Dimension}\ErrorTok{(}\ExtensionTok{CartanSubalgebra}\ErrorTok{(}\ExtensionTok{h}\KeywordTok{));}
\ExtensionTok{12}
\end{Highlighting}
\end{Shaded}

Notice: we found some minor misspellings in the code:

\begin{itemize}
\item
  \texttt{"D",4,5} is \(\mathfrak{so}(1,7)\),
\item
  \texttt{"D",4,4} is \(\mathfrak{so}(3,5)\),
\item
  \texttt{"E",7,3} is \(\mathfrak{e}_{7(-25)}=EVII\),
\item
  \texttt{"E",7,4} is \(\mathfrak{e}_{7(-5)}=EVI\).
\end{itemize}

Compare the real rank and the dimension of the real form with the data
given in Table 4 in (\protect\hyperlink{ref-onvin}{Onishchik and Vinberg
1990}).

To make our work easier, we wrote our own function to recognize the
triples.

\begin{verbatim}
GetSymbolSimple(type, rank, id)
\end{verbatim}

This function works up to \texttt{rank} 8 and returns string
corresponding to the symbol of real simple Lie algebra. Remark: we use
notation from (\protect\hyperlink{ref-helgason}{Helgason 2001}).

Example:

\begin{Shaded}
\begin{Highlighting}[]
\ExtensionTok{gap}\OperatorTok{\textgreater{}}\NormalTok{ GetSymbolSimple}\ErrorTok{(}\StringTok{"C"}\ExtensionTok{,4,4}\KeywordTok{);}
\StringTok{"sp(4,R)"}
\ExtensionTok{gap}\OperatorTok{\textgreater{}}\NormalTok{ GetSymbolSimple}\ErrorTok{(}\StringTok{"A"}\ExtensionTok{,5,4}\KeywordTok{);}
\StringTok{"su(3,3)"}
\ExtensionTok{gap}\OperatorTok{\textgreater{}}\NormalTok{ GetSymbolSimple}\ErrorTok{(}\StringTok{"A"}\ExtensionTok{,5,5}\KeywordTok{);}
\StringTok{"su*(6)"}
\end{Highlighting}
\end{Shaded}

\hypertarget{semisimple-real-lie-algebra}{%
\subsubsection{Semisimple real Lie
algebra}\label{semisimple-real-lie-algebra}}

We propose the following convention to represent a semisimple real Lie
algebra of rank up to 8. Due to the limitations of GAP (see
\href{https://www.gap-system.org/Faq/faq.html\#6.2}{this}), we represent
a semisimple real Lie algebra \(\mathfrak{g}\) as tuple:

\begin{verbatim}
[real rank, non-compact dimension,type1,rank1,id1,...,typen,rankn,idn]
\end{verbatim}

where \texttt{type1,\ rank1,id1} \ldots{} \texttt{typen,rankn,idn}
represent the corresponding triples of simple components of
\(\mathfrak{g}\) (we use the fact that a semisimple algebra is a direct
sum of simple algebras). The first two coordinates in the input vector
are given to reduce the complexity of the algorithm.

\begin{verbatim}
GetSymbolSemisimple(tuple)
\end{verbatim}

The input is a tuple corresponding to the semisimple real Lie algebra
\(\mathfrak{g}\) (represented in our notation). This function returns
the symbol of \(\mathfrak{g}\).

Example:

\begin{Shaded}
\begin{Highlighting}[]
\ExtensionTok{gap}\OperatorTok{\textgreater{}}\NormalTok{ T:=[ 4, 10, }\StringTok{"A"}\NormalTok{, 1, 2, }\StringTok{"A"}\NormalTok{, 1, 2, }\StringTok{"B"}\NormalTok{, 2, 2 ]}\KeywordTok{;}
\BuiltInTok{[}\NormalTok{ 4, 10, }\StringTok{"A"}\NormalTok{, }\ErrorTok{1,}\NormalTok{ 2, }\StringTok{"A"}\NormalTok{, 1, 2, }\StringTok{"B"}\NormalTok{, 2, 2 ]}
\ExtensionTok{gap}\OperatorTok{\textgreater{}}\NormalTok{ GetSymbolSemisimple}\ErrorTok{(}\ExtensionTok{T}\KeywordTok{);}
\StringTok{"sl(2,R)+sl(2,R)+so(2,3)"}
\end{Highlighting}
\end{Shaded}

If you are familiar with CoReLG Package
(\protect\hyperlink{ref-CoReLG}{Dietrich, Faccin, and Graaf 2014}) and
want to work with Lie algebra object, we have created the following
function:

\begin{verbatim}
RealFormByTuple(tuple)
\end{verbatim}

The input is a tuple corresponding to a semisimple real Lie algebra
\(\mathfrak{g}\) (represented in our notation). This function returns a
Lie algebra object corresponding to \(\mathfrak{g}\).

Example:

\begin{Shaded}
\begin{Highlighting}[]
\ExtensionTok{gap}\OperatorTok{\textgreater{}}\NormalTok{ T:=[ 4, 10, }\StringTok{"A"}\NormalTok{, 1, 2, }\StringTok{"A"}\NormalTok{, 1, 2, }\StringTok{"B"}\NormalTok{, 2, 2 ]}\KeywordTok{;}
\BuiltInTok{[}\NormalTok{ 4, 10, }\StringTok{"A"}\NormalTok{, }\ErrorTok{1,}\NormalTok{ 2, }\StringTok{"A"}\NormalTok{, 1, 2, }\StringTok{"B"}\NormalTok{, 2, 2 ]}
\ExtensionTok{gap}\OperatorTok{\textgreater{}}\NormalTok{ GetSymbolSemisimple}\ErrorTok{(}\ExtensionTok{T}\KeywordTok{);}
\StringTok{"sl(2,R)+sl(2,R)+so(2,3)"}
\ExtensionTok{gap}\OperatorTok{\textgreater{}}\NormalTok{ g:=RealFormByTuple}\ErrorTok{(}\ExtensionTok{T}\KeywordTok{);}
\OperatorTok{\textless{}}\NormalTok{Lie }\ExtensionTok{algebra}\NormalTok{ of dimension 16 over SqrtField}\OperatorTok{\textgreater{}}
\ExtensionTok{gap}\OperatorTok{\textgreater{}}\NormalTok{ RealRank}\ErrorTok{(}\ExtensionTok{g}\KeywordTok{);}
\ExtensionTok{4}
\ExtensionTok{gap}\OperatorTok{\textgreater{}}\NormalTok{ NonCompactDimension}\ErrorTok{(}\ExtensionTok{g}\KeywordTok{);}
\ExtensionTok{10}
\end{Highlighting}
\end{Shaded}

The reverse operation is ineffective in GAP. The conversion takes a very
long time. Often the program closes without warning. The only option is
to manually search for the appropriate isomorphisms.

Example:

\begin{Shaded}
\begin{Highlighting}[]
\ExtensionTok{gap}\OperatorTok{\textgreater{}}\NormalTok{ g:=RealFormById}\ErrorTok{(}\StringTok{"E"}\ExtensionTok{,6,3}\KeywordTok{);}
\OperatorTok{\textless{}}\NormalTok{Lie }\ExtensionTok{algebra}\NormalTok{ of dimension 78 over SqrtField}\OperatorTok{\textgreater{}}
\ExtensionTok{gap}\OperatorTok{\textgreater{}}\NormalTok{ GetSymbolSimple}\ErrorTok{(}\StringTok{"E"}\ExtensionTok{,6,3}\KeywordTok{);}
\StringTok{"E6(2)"}
\ExtensionTok{gap}\OperatorTok{\textgreater{}}\NormalTok{ k:=CartanDecomposition}\ErrorTok{(}\ExtensionTok{G}\KeywordTok{)}\ExtensionTok{.K}\KeywordTok{;}
\OperatorTok{\textless{}}\NormalTok{Lie }\ExtensionTok{algebra}\NormalTok{ of dimension 38 over SqrtField}\OperatorTok{\textgreater{}}
\ExtensionTok{gap}\OperatorTok{\textgreater{}}\NormalTok{ T:=[0,0,}\StringTok{"A"}\NormalTok{,1,1,}\StringTok{"A"}\NormalTok{,5,1]}\KeywordTok{;}
\BuiltInTok{[}\NormalTok{ 0, 0, }\StringTok{"A"}\NormalTok{, }\ErrorTok{1,}\NormalTok{ 1, }\StringTok{"A"}\NormalTok{, 5, 1 ]}
\ExtensionTok{gap}\OperatorTok{\textgreater{}}\NormalTok{ GetSymbolSemisimple}\ErrorTok{(}\ExtensionTok{T}\KeywordTok{);}
\StringTok{"su(2)+su(6)"}
\ExtensionTok{gap}\OperatorTok{\textgreater{}}\NormalTok{ l:=RealFormByTuple}\ErrorTok{(}\ExtensionTok{T}\KeywordTok{);}
\OperatorTok{\textless{}}\NormalTok{Lie }\ExtensionTok{algebra}\NormalTok{ of dimension 38 over SqrtField}\OperatorTok{\textgreater{}}
\ExtensionTok{gap}\OperatorTok{\textgreater{}}\NormalTok{ IsomorphismOfRealSemisimpleLieAlgebras}\ErrorTok{(}\ExtensionTok{k,l}\KeywordTok{);}
\OperatorTok{\textless{}}\NormalTok{Lie }\ExtensionTok{algebra}\NormalTok{ isomorphism between Lie algebras of dimension 38 over SqrtField}\OperatorTok{\textgreater{}}
\end{Highlighting}
\end{Shaded}

\begin{verbatim}
CheckTuple(tuple)
\end{verbatim}

The inupt is a tuple. This function returns ``true'' when the tuple
corresponds to a real semisimple Lie algebra (it works only up to 8
simple components).

\begin{Shaded}
\begin{Highlighting}[]
\ExtensionTok{gap}\OperatorTok{\textgreater{}}\NormalTok{ T:=[0,0,}\StringTok{"A"}\NormalTok{,1,1,}\StringTok{"A"}\NormalTok{,5,1]}\KeywordTok{;}
\BuiltInTok{[}\NormalTok{ 0, 0, }\StringTok{"A"}\NormalTok{, }\ErrorTok{1,}\NormalTok{ 1, }\StringTok{"A"}\NormalTok{, 5, 1 ]}
\ExtensionTok{gap}\OperatorTok{\textgreater{}}\NormalTok{ CheckTuple}\ErrorTok{(}\ExtensionTok{T}\KeywordTok{);}
\FunctionTok{true}
\ExtensionTok{gap}\OperatorTok{\textgreater{}}\NormalTok{ T:=[0,2,}\StringTok{"A"}\NormalTok{,1,1,}\StringTok{"A"}\NormalTok{,5,1]}\KeywordTok{;}
\BuiltInTok{[}\NormalTok{ 0, 2, }\StringTok{"A"}\NormalTok{, }\ErrorTok{1,}\NormalTok{ 1, }\StringTok{"A"}\NormalTok{, 5, 1 ]}
\ExtensionTok{gap}\OperatorTok{\textgreater{}}\NormalTok{ CheckTuple}\ErrorTok{(}\ExtensionTok{T}\KeywordTok{);}
\FunctionTok{false}
\end{Highlighting}
\end{Shaded}

\hypertarget{functions-for-complex-lie-algebra}{%
\subsection{Functions for complex Lie
algebra}\label{functions-for-complex-lie-algebra}}

\begin{verbatim}
NicerSemisimpleSubalgebras(typeRank)
\end{verbatim}

The function is based on \texttt{LieAlgebraAndSubalgebras} function from
SLA Package (\protect\hyperlink{ref-sla}{Graaf 2018}). It returns a list
of types of semisimple subalgebras of a complex Lie algebra (for a given
type and rank). Remark: some types may correspond to several linearly
equivalence subalgebras.

\begin{verbatim}
NumberOfSubalgebraClasses(typeRankG, typeRankH)
\end{verbatim}

This function returns the number of linear equivalence classes for a
given type of subalgebra (which is the value of the second argument) in
complex Lie algebra (which is the value of the first argument).

Example:

\begin{Shaded}
\begin{Highlighting}[]
\ExtensionTok{gap}\OperatorTok{\textgreater{}}\NormalTok{ NicerSemisimpleSubalgebras}\ErrorTok{(}\StringTok{"E6"}\KeywordTok{);}
\BuiltInTok{[} \StringTok{"A1"}\NormalTok{, }\StringTok{"A2"}\NormalTok{, }\StringTok{"B2"}\NormalTok{, }\ErrorTok{"G2",} \StringTok{"A1 A1"}\NormalTok{, }\StringTok{"A1 A1 A1"}\NormalTok{, }\StringTok{"A1 A2"}\NormalTok{, }\StringTok{"A1 B2"}\NormalTok{, }\StringTok{"A3"}\NormalTok{, }\StringTok{"A1 G2"}\NormalTok{, }\StringTok{"B3"}\NormalTok{, }
\StringTok{"C3"}\ExtensionTok{,} \StringTok{"A1 A1 A1 A1"}\NormalTok{, }\StringTok{"A1 A1 A2"}\NormalTok{, }\StringTok{"A1 A1 B2"}\NormalTok{, }\StringTok{"A2 A2"}\NormalTok{, }\StringTok{"A1 A3"}\NormalTok{, }\StringTok{"B2 B2"}\NormalTok{, }\StringTok{"A2 G2"}\NormalTok{, }\StringTok{"A1 C3"}\NormalTok{,}
\StringTok{"A1 B3"}\ExtensionTok{,} \StringTok{"A4"}\NormalTok{, }\StringTok{"D4"}\NormalTok{, }\StringTok{"C4"}\NormalTok{, }\StringTok{"B4"}\NormalTok{, }\StringTok{"F4"}\NormalTok{, }\StringTok{"A1 A2 A2"}\NormalTok{, }\StringTok{"A1 A1 A3"}\NormalTok{,  }\StringTok{"A1 A4"}\NormalTok{, }\StringTok{"A5"}\NormalTok{, }\StringTok{"D5"}\NormalTok{, }
\StringTok{"A2 A2 A2"}\ExtensionTok{,} \StringTok{"A1 A5"}\NormalTok{ ]}
\ExtensionTok{gap}\OperatorTok{\textgreater{}}\NormalTok{ NumberOfSubalgebraClasses}\ErrorTok{(}\StringTok{"E6"}\ExtensionTok{,}\StringTok{"A1 A1"}\KeywordTok{);}
\ExtensionTok{24}
\ExtensionTok{gap}\OperatorTok{\textgreater{}}\NormalTok{ NumberOfSubalgebraClasses}\ErrorTok{(}\StringTok{"E6"}\ExtensionTok{,}\StringTok{"A5"}\KeywordTok{);}
\ExtensionTok{1}
\end{Highlighting}
\end{Shaded}

\hypertarget{function-for-real-lie-algebras}{%
\subsection{Function for real Lie
algebras}\label{function-for-real-lie-algebras}}

\begin{verbatim}
RealRank(g)
\end{verbatim}

The input is a real Lie algebra \(\mathfrak{g}\) (as an object, not as a
tuple). The output is the real rank of \(\mathfrak{g}\) (the dimension
of the Cartan subalgebra of \(\mathfrak{g}\)).

\begin{verbatim}
NonCompactDimension(g)
\end{verbatim}

The input is a real Lie algebra \(\mathfrak{g}\). This function returns
non-compact dimension of \(\mathfrak{g}\) (dimension of non-compact part
from the Cartan decomposition for \(\mathfrak{g}\)).

\begin{verbatim}
CompactDimension(g)
\end{verbatim}

The input is a real Lie algebra \(\mathfrak{g}\). This function returns
the compact dimension of \(\mathfrak{g}\) (dimension of the compact part
of the Cartan decomposition for \(\mathfrak{g}\)).

Example:

\begin{Shaded}
\begin{Highlighting}[]
\ExtensionTok{gap}\OperatorTok{\textgreater{}}\NormalTok{ g:=RealFormById}\ErrorTok{(}\StringTok{"A"}\ExtensionTok{,5,3}\KeywordTok{);}
\OperatorTok{\textless{}}\NormalTok{Lie }\ExtensionTok{algebra}\NormalTok{ of dimension 35 over SqrtField}\OperatorTok{\textgreater{}}
\ExtensionTok{gap}\OperatorTok{\textgreater{}}\NormalTok{ RealRank}\ErrorTok{(}\ExtensionTok{g}\KeywordTok{);}
\ExtensionTok{2}
\ExtensionTok{gap}\OperatorTok{\textgreater{}}\NormalTok{ NonCompactDimension}\ErrorTok{(}\ExtensionTok{g}\KeywordTok{);}
\ExtensionTok{16}
\ExtensionTok{gap}\OperatorTok{\textgreater{}}\NormalTok{ CompactDimension}\ErrorTok{(}\ExtensionTok{g}\KeywordTok{);}
\ExtensionTok{19}
\ExtensionTok{gap}\OperatorTok{\textgreater{}}\NormalTok{ T:=[ 5, 19, }\StringTok{"A"}\NormalTok{, 1, 2, }\StringTok{"A"}\NormalTok{, 1, 2, }\StringTok{"A"}\NormalTok{, 1, 1, }\StringTok{"D"}\NormalTok{, 4, 4 ]}\KeywordTok{;}
\BuiltInTok{[}\NormalTok{ 5, 19, }\StringTok{"A"}\NormalTok{, }\ErrorTok{1,}\NormalTok{ 2, }\StringTok{"A"}\NormalTok{, 1, 2, }\StringTok{"A"}\NormalTok{, 1, 1, }\StringTok{"D"}\NormalTok{, 4, 4 ]}
\ExtensionTok{gap}\OperatorTok{\textgreater{}}\NormalTok{ g:=RealFormByTuple}\ErrorTok{(}\ExtensionTok{T}\KeywordTok{);}
\OperatorTok{\textless{}}\NormalTok{Lie }\ExtensionTok{algebra}\NormalTok{ of dimension 37 over SqrtField}\OperatorTok{\textgreater{}}
\ExtensionTok{gap}\OperatorTok{\textgreater{}}\NormalTok{ RealRank}\ErrorTok{(}\ExtensionTok{g}\KeywordTok{);}
\ExtensionTok{5}
\ExtensionTok{gap}\OperatorTok{\textgreater{}}\NormalTok{ NonCompactDimension}\ErrorTok{(}\ExtensionTok{g}\KeywordTok{);}
\ExtensionTok{19}
\ExtensionTok{gap}\OperatorTok{\textgreater{}}\NormalTok{ CompactDimension}\ErrorTok{(}\ExtensionTok{g}\KeywordTok{);}
\ExtensionTok{18}
\end{Highlighting}
\end{Shaded}

\hypertarget{function-for-generating-potential-subalgebras-and-subalgebra-pairs}{%
\subsection{Function for generating potential subalgebras and subalgebra
pairs}\label{function-for-generating-potential-subalgebras-and-subalgebra-pairs}}

The following features are described in more detail in our article
(\protect\hyperlink{ref-BJT}{Bocheński, Jastrzębski, and Tralle 2019}).

\begin{verbatim}
PotentialSubalgebras("type",rank,id)
\end{verbatim}

The input is a triple corresponding to a simple real Lie algebra
\(\mathfrak{g}\) of rank up to 8. This function returns a list of
semisimple subalgebras in \(\mathfrak{g}\) as described in Algorithm 1
(\protect\hyperlink{ref-BJT}{Bocheński, Jastrzębski, and Tralle 2019}).

\begin{verbatim}
PotentialSubalgebraPairs("type",rank,id)
\end{verbatim}

The input is a triple corresponding to a simple real Lie algebra
\(\mathfrak{g}\) of rank up to 8. This function returns subalgebra pairs
in \(\mathfrak{g}\) as described in Algorithm 2
(\protect\hyperlink{ref-BJT}{Bocheński, Jastrzębski, and Tralle 2019}).

\begin{verbatim}
GetSymbolPairs(output of previous function)
\end{verbatim}

This function returns symbols of the triples (see example below).

\begin{Shaded}
\begin{Highlighting}[]
\ExtensionTok{gap}\OperatorTok{\textgreater{}}\NormalTok{ L:=PotentialSubalgebraPairs}\ErrorTok{(}\StringTok{"E"}\ExtensionTok{,6,2}\KeywordTok{);}
\ExtensionTok{Counting:}\NormalTok{ 30\%}
\ExtensionTok{Counting:}\NormalTok{ 60\%}
\ExtensionTok{Counting:}\NormalTok{ 90\%}
\ExtensionTok{Counting}\NormalTok{ completed.}
\BuiltInTok{[}\NormalTok{ [ [ 2, }\ErrorTok{14,} \StringTok{"B"}\NormalTok{, 4, 2 ], [ 4, 28, }\StringTok{"F"}\NormalTok{, 4, 2 ] ], [ [ 2, 14, }\StringTok{"A"}\NormalTok{, 5, 5 ], }
\BuiltInTok{[}\NormalTok{ 4, 28, }\StringTok{"F"}\NormalTok{, }\ErrorTok{4,}\NormalTok{ 2 ] ], [ [ 3, 21, }\StringTok{"D"}\NormalTok{, 5, 6 ], [ 3, 21, }\StringTok{"D"}\NormalTok{, 5, 6 ] ],}
\BuiltInTok{[}\NormalTok{ [ 2, 14, }\ErrorTok{"A",}\NormalTok{ 1, 1, }\StringTok{"A"}\NormalTok{, 5, 5 ], [ 4, 28, }\StringTok{"F"}\NormalTok{, 4, 2 ] ] ]}
\ExtensionTok{gap}\OperatorTok{\textgreater{}}\NormalTok{ GetSymbolPairs}\ErrorTok{(}\ExtensionTok{L}\KeywordTok{);}
\VariableTok{h=}\NormalTok{so}\KeywordTok{(}\ExtensionTok{2,7}\KeywordTok{)}\ExtensionTok{,}\NormalTok{ l=F4}\ErrorTok{(}\ExtensionTok{4}\KeywordTok{)}
\VariableTok{h=}\NormalTok{su}\PreprocessorTok{*(}\NormalTok{6}\PreprocessorTok{)}\NormalTok{, }\VariableTok{l=}\NormalTok{F4}\KeywordTok{(}\ExtensionTok{4}\KeywordTok{)}
\VariableTok{h=}\NormalTok{so}\KeywordTok{(}\ExtensionTok{3,7}\KeywordTok{)}\ExtensionTok{,}\NormalTok{ l=so}\ErrorTok{(}\ExtensionTok{3,7}\KeywordTok{)}
\VariableTok{h=}\NormalTok{su}\KeywordTok{(}\ExtensionTok{2}\KeywordTok{)}\ExtensionTok{+su*}\ErrorTok{(}\ExtensionTok{6}\KeywordTok{)}\ExtensionTok{,}\NormalTok{ l=F4}\ErrorTok{(}\ExtensionTok{4}\KeywordTok{)}
\ExtensionTok{gap}\OperatorTok{\textgreater{}}\NormalTok{ GetSymbolSimple}\ErrorTok{(}\StringTok{"E"}\ExtensionTok{,6,2}\KeywordTok{);}
\StringTok{"E6(6)"}
\end{Highlighting}
\end{Shaded}

\hypertarget{references}{%
\section*{References}\label{references}}
\addcontentsline{toc}{section}{References}

\hypertarget{refs}{}
\begin{CSLReferences}{1}{0}
\leavevmode\hypertarget{ref-BJT}{}%
Bocheński, M., P. Jastrzębski, and A. Tralle. 2019. {``Non-Existence of
Standard Compact Clifford-Klein Forms of Homogeneous Spaces of
Exceptional Lie Groups.''}

\leavevmode\hypertarget{ref-CoReLG}{}%
Dietrich, H., P. Faccin, and W. A. de Graaf. 2014. {``CoReLG,
Computation with Real Lie Groups, Version 1.20.''}
\url{http://users.monash.edu/~heikod/corelg/}.

\leavevmode\hypertarget{ref-sla}{}%
Graaf, W. A. de. 2018. {``SLA, Simple Lie Algebras, Version 1.5.''}
\url{http://www.science.unitn.it/~degraaf/sla.html}.

\leavevmode\hypertarget{ref-helgason}{}%
Helgason, S. 2001. \emph{Differential Geometry and Symmetric Spaces}.
American Mathematical Society.

\leavevmode\hypertarget{ref-onvin}{}%
Onishchik, A., and E. Vinberg. 1990. \emph{Lie Groups and Algebraic
Groups}. Springer Series in Soviet Mathematics. Springer-Verlag Berlin
Heidelberg.

\end{CSLReferences}

\end{document}
